\begin{abstract}
	Ein Abstract ist ein Text, der die Leserinnen und Leser in kompakter Form über den Inhalt einer wissenschaftlichen Arbeit informiert. Es soll in gewisser Weise Überzeugungsarbeit leisten.
	
	Im Falle der VWA sollte das Abstract 1.000 bis 1.500 Zeichen (inkl. Leerzeichen) umfassen und an den Beginn gestellt werden. Das Abstract folgt bei der VWA direkt nach dem Titelblatt, steht also vor dem (optionalen) Vorwort und dem Inhaltsverzeichnis. Ihm wird kein eigener Gliederungspunkt zugewiesen. Das erste mit einer Ziffer versehene Kapitel ist die Einleitung („1 Einleitung“).
	NICHTS PERSÖNLICHES IN DEN ABSTRACT SCHREIBEN! (Ein Abstract ist keine Danksagung!)
	
	\textbf{Woraus sollte ein Abstract bestehen?}
	
	\begin{enumerate}
		\item Verortung im Forschungsfeld: Wie ist der derzeitige Forschungsstand? Warum ist dieses Thema von gesellschaftlichem Interesse? (1-2 Sätze)
		\item Aufzeigen der Forschungsnische: Welche Forschungslücke schließt diese Arbeit? (1-2 Sätze)
		\item Besetzen der Forschungsnische (der ''Rest'')
	\end{enumerate}
\end{abstract}